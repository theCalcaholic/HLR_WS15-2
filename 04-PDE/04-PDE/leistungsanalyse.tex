\documentclass[fleqn]{article}
\usepackage[x11names, rgb]{xcolor}
\usepackage[utf8]{inputenc}
\usepackage{tikz}
\usepackage{geometry}
\usepackage{fancyhdr}
\usepackage{amsmath,amsthm,amssymb}
\usepackage{graphicx}
\usepackage{hyperref}
\usepackage{lipsum}
\usepackage{ulem}
\usepackage{comment}
\usepackage{enumerate}
\usepackage{titlesec}
\usepackage{boolexpr,pdftexcmds,trace}
\usepackage{pgfplotstable}
\usepackage{standalone}
\makeatletter

\usetikzlibrary{snakes,arrows,shapes}
\newwrite\dotfile

\begingroup
  \catcode`\[ = 1\relax
  \catcode`\] = 2\relax
  \catcode`\{ = 12\relax
  \catcode`\} = 12 \relax
  \gdef\OpenBrace[{]
  \gdef\CloseBrace[}]
\endgroup

% custom commands
\newcommand{\leadingzero}[1]{\ifnum #1<10 0#1\else#1\fi}
\newcommand{\gerdate}[3]{\leadingzero{#1}.\leadingzero{#2}.\leadingzero{#3}}
\newcommand{\gertoday}{\gerdate{\the\day}{\the\month}{\the\year}}
\newcommand*{\bfrac}[2]{\genfrac{}{}{0pt}{}{#1}{#2}}
\newcommand{\R}{\mathbb{R}}
\newcommand{\N}{\mathbb{N}}
\newcommand{\Q}{\mathbb{Q}}
\newcommand{\Z}{\mathbb{Z}}
\newcommand{\dotarrow}[0]{}

\newenvironment{graphviz}[1]%
{%
\switch
\case{\pdf@strcmp{#1}{graph}}
    \renewcommand{\dotarrow}[0]{--}
\case{\pdf@strcmp{#1}{strict graph}}
    \renewcommand{\dotarrow}[0]{--}
\case{\pdf@strcmp{#1}{digraph}}
    \renewcommand{\dotarrow}[0]{->}
\case{\pdf@strcmp{#1}{strict digraph}}
    \renewcommand{\dotarrow}[0]{->}
\endswitch

\immediate\openout\dotfile=tmp.dot%
\newcommand{\node}[2]{%
\immediate\write\dotfile{##1 \dotarrow \OpenBrace##2\CloseBrace}%
}%
%
\immediate\write\dotfile{#1 \OpenBrace}
}%
{\immediate\write\dotfile{\CloseBrace}%
\immediate\closeout\dotfile%
\immediate\write18{dot2tex --figonly tmp.dot > tmp.tex}%
\input{tmp.tex}%
}

\setcounter{section}{0}
\setcounter{subsection}{0}
\pagestyle{fancy}

\lhead{Tobias Knöppler, Klaus Ziegert}
\chead{}
\rhead{\gertoday}
\lfoot{}
\cfoot{\thepage}
\rfoot{}
%\setlength{\mathindent}{0pt}

% document specific settings
%\renewcommand{\thesection}{}
\renewcommand{\thesubsection}{\arabic{section}. \alph{subsection})}
\renewcommand{\thesubsubsection}{\roman{subsubsection})}
\titleformat{\subsubsection}[runin]{\normalfont\normalsize\bfseries}{\thesubsubsection}{1em}{}

\title{Leistungsanalyse}
%\author{}
%\date{\gertoday}
\begin{document}
\section{variable Threadzahl}
\begin{tikzpicture}
  \begin{axis}[
    xlabel=Threads,
    ylabel=Berechnungsdauer,
    xmin=0,
    ymin=0,
    xmax=12,
    ymax=200,
    width=\textwidth
  ]
    \addplot[color=red, mark=*] 
        table[x=Threads, y=Berechnungsdauer]{pgfplots.performance_threads_run1.dat};
    \addplot[color=blue, mark=*] 
        table[x=Threads, y=Berechnungsdauer]{pgfplots.performance_threads_run2.dat};
    \addplot[color=green, mark=*] 
        table[x=Threads, y=Berechnungsdauer]{pgfplots.performance_threads_run3.dat};
    \legend{Testrun 1, Testrun 2, Testrun 3}
  \end{axis}
\end{tikzpicture}

\section{variable Anzahl Interlines}

\begin{tikzpicture}
  \begin{axis}[
    xlabel=Interlines,
    ylabel=Berechnungsdauer,
    xmin=0,
    ymin=0,
    xmax=1024,
    ymax=800,
    width=\textwidth
  ]
    \addplot[color=red, mark=*]
        table[x=Interlines, y=Berechnungsdauer]{pgfplots.performance_interlines_run1.dat};
    \addplot[color=blue, mark=*]
        table[x=Interlines, y=Berechnungsdauer]{pgfplots.performance_interlines_run2.dat};
    \addplot[color=green, mark=*]
        table[x=Interlines, y=Berechnungsdauer]{pgfplots.performance_interlines_run3.dat};
    \legend{Testrun 1, Testrun 2, Testrun 3}
  \end{axis}
\end{tikzpicture}

\end{document}
