\documentclass[fleqn]{article}
\usepackage[x11names, rgb]{xcolor}
\usepackage[utf8]{inputenc}
\usepackage{tikz}
\usepackage{geometry}
\usepackage{fancyhdr}
\usepackage{amsmath,amsthm,amssymb}
\usepackage{graphicx}
\usepackage{hyperref}
\usepackage{lipsum}
\usepackage{ulem}
\usepackage{comment}
\usepackage{enumerate}
\usepackage{titlesec}
\usepackage{boolexpr,pdftexcmds,trace}
\usepackage{pgfplotstable}
\usepackage{standalone}
\makeatletter

\pgfplotsset{compat=1.11}
\usepgfplotslibrary{fillbetween}

\usetikzlibrary{snakes,arrows,shapes}
\newwrite\dotfile

\begingroup
  \catcode`\[ = 1\relax
  \catcode`\] = 2\relax
  \catcode`\{ = 12\relax
  \catcode`\} = 12 \relax
  \gdef\OpenBrace[{]
  \gdef\CloseBrace[}]
\endgroup

% custom commands
\newcommand{\leadingzero}[1]{\ifnum #1<10 0#1\else#1\fi}
\newcommand{\gerdate}[3]{\leadingzero{#1}.\leadingzero{#2}.\leadingzero{#3}}
\newcommand{\gertoday}{\gerdate{\the\day}{\the\month}{\the\year}}
\newcommand*{\bfrac}[2]{\genfrac{}{}{0pt}{}{#1}{#2}}
\newcommand{\R}{\mathbb{R}}
\newcommand{\N}{\mathbb{N}}
\newcommand{\Q}{\mathbb{Q}}
\newcommand{\Z}{\mathbb{Z}}
\newcommand{\dotarrow}[0]{}

\newenvironment{graphviz}[1]%
{%
\switch
\case{\pdf@strcmp{#1}{graph}}
    \renewcommand{\dotarrow}[0]{--}
\case{\pdf@strcmp{#1}{strict graph}}
    \renewcommand{\dotarrow}[0]{--}
\case{\pdf@strcmp{#1}{digraph}}
    \renewcommand{\dotarrow}[0]{->}
\case{\pdf@strcmp{#1}{strict digraph}}
    \renewcommand{\dotarrow}[0]{->}
\endswitch

\immediate\openout\dotfile=tmp.dot%
\newcommand{\node}[2]{%
\immediate\write\dotfile{##1 \dotarrow \OpenBrace##2\CloseBrace}%
}%
%
\immediate\write\dotfile{#1 \OpenBrace}
}%
{\immediate\write\dotfile{\CloseBrace}%
\immediate\closeout\dotfile%
\immediate\write18{dot2tex --figonly tmp.dot > tmp.tex}%
\input{tmp.tex}%
}

\setcounter{section}{0}
\setcounter{subsection}{0}
\pagestyle{fancy}

\lhead{Tobias Knöppler, Klaus Ziegert}
\chead{}
\rhead{\gertoday}
\lfoot{}
\cfoot{\thepage}
\rfoot{}
%\setlength{\mathindent}{0pt}

% document specific settings
%\renewcommand{\thesection}{}
%\renewcommand{\thesubsection}{\arabic{subsection})}
%\renewcommand{\thesubsubsection}{\roman{subsubsection})}
%\titleformat{\subsubsection}[runin]{\normalfont\normalsize\bfseries}{\thesubsubsection}{1em}{}

\title{Leistungsanalyse}
%\author{}
%\date{\gertoday}
\begin{document}
\section{variable Threadzahl}
\begin{tikzpicture}
  \begin{axis}[
    xlabel=Threads,
    ylabel=Berechnungsdauer (s),
    xmin=0,
    ymin=0,
    xmax=12,
    ymax=200,
    width=\textwidth
  ]
%    \path[name path=xaxis] (axis cs:1, 0) -- (axis cs:0, 0);
    %\addplot[color=red, mark=*] 
    %    table[x=Threads, y=Berechnungsdauer]{pgfplots.performance_threads_run1.dat};
    \addplot[color=blue, mark=*] 
        table[x=Threads, y=Berechnungsdauer]{pgfplots.performance_threads_run2.dat};
    \addplot[color=green, mark=*] 
        table[x=Threads, y=Berechnungsdauer]{pgfplots.performance_threads_run3.dat};
    \addplot[color=purple, mark=*] 
        table[x=Threads, y=Berechnungsdauer]{pgfplots.performance_threads_run4.dat};
    \legend{Testrun 1, Testrun 2, Testrun 3, ideal}
    \addplot[name path=perfectF,
        thick,
        color=yellow, mark=x]
        table[x=Threads, y=Berechnungsdauer]{pgfplots.performance_threads_perfect.dat};
%    \addplot[
%      thick,:x
%      color=yellow,
%      fill=yellow,
%      fill opacity=1]
%    fill between[
%      of=perfectF and xaxis,
%      soft clip={domain=0:1},
%    ];
  \end{axis}
\end{tikzpicture}

\subsection{Interpretation:}
Eine Verdoppelung der Threadzahl halbiert bis zu dem Schritt von 5 auf 10 Prozessoren die Berechnungszeit einigermaßen genau, d.h. es ist uns gelungen, die verwendete Hardware bei diesem Problem bis zu einer Verteilung auf 12 Akteure (Kerne) sehr effizient auszunutzen.Erst beim der Verwendung von mehr als 8. Kernen weicht der Leistungszuwachs deutlich von der idealen Funktion (Berechnungszeit mit einem Kern = $\frac{1}{x} /cdot $ Berechnungszeit mit x Kernen, hier eingezeichnet in gelb) ab.
Jedoch ist die Verdopplung der verfügbaren Hardware natürlich ab gewissen Grenzen sehr teuer.



\section{variable Anzahl Interlines}

\begin{tikzpicture}
  \begin{axis}[
    xlabel=Interlines,
    ylabel=Berechnungsdauer (s),
    xmin=0,
    ymin=0,
    xmax=1024,
    ymax=800,
    width=\textwidth
  ]
    \addplot[color=red, mark=*]
        table[x=Interlines, y=Berechnungsdauer]{pgfplots.performance_interlines_run1.dat};
    \addplot[color=blue, mark=*]
        table[x=Interlines, y=Berechnungsdauer]{pgfplots.performance_interlines_run2.dat};
    \addplot[color=green, mark=*]
        table[x=Interlines, y=Berechnungsdauer]{pgfplots.performance_interlines_run3.dat};
    \legend{Testrun 1, Testrun 2, Testrun 3}
  \end{axis}
\end{tikzpicture}

\subsection{Interpretation:}
Man kann anhand dieser Skizze erkennen, dass der Wachstum der
Berechnungszeit ungefähr polynomial steigt. Über die Anfangswerte an
Interlines (1- 16) kann man keine gute Aussage treffen, da die
verschiedenen Tests ziemlich stark voneinander ausfallen. Erst ab 16
Interlines verdreifacht sich die Ausführung bei Quadrierung der
Interlines. Dieser Faktor zwischen  $2^{n}$ und $2^{n+1}$ steigt
ungefähr linear an. und hat entsprechend bei $n=9$ den größten Wert.
Dieses Ergebnis war ungefähr zu erwarten, da die bei Hinzunahme von
Interlines pontenziale Berechnungseinheiten mit sich zieht. 
 



\end{document}
