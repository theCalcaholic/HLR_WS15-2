\documentclass[a4paper,11pt]{scrartcl}
\usepackage[utf8]{inputenc}
\usepackage{amsmath}
\usepackage{amssymb}
\usepackage{dsfont}
\usepackage{pgfplots}
\pagestyle{empty}
\title{Leistungsanalyse}
\author{Klaus-Johan Ziegert \and Tobias Knoppler}

\begin{document}
\maketitle
\section{Partdiff-seq Data}
Hier wurde die Funktion Partdiff-seq mit 512 Interlines und 65 Iterations aufgerufen.\\
\\
\begin{tabular}{|c|c|c|c|}            \hline
Anzahl der Threads \textbackslash Test & 1 & 2 & 3 \\ \hline
1 & 52.018536 &     51.960722 &  51.954834\\
\hline
\end{tabular}

\section{Partdiff-posix Data}

Hier wurde die Funktion Partdiff-posix mit 512 Interlines und 65 Iterations aufgerufen.\\
\\

\begin{tabular}{|c|c|c|c|}            \hline
Anzahl der Threads \textbackslash Test & 1 & 2 & 3 \\ \hline
1 & 52.028838 &     52.140695 &     52.938863\\
2  & 26.087554 &     26.821407 &      26.110185\\
3   & 17.549110 &     17.515558 &     17.551143\\
4   & 13.151488 &     13.143059  &    13.140707\\
5  & 10.686832 &     10.544150  &    10.509162 \\
6 &  8.777090  &    8.838507  &    8.783395  \\
7  & 7.667535  &    7.544237 &     7.536613  \\
8  &  6.610692 &     9.539265 &     9.666401 \\
9  &  5.894046 &     6.146108 &     6.024277 \\
10  &  7.688474  &    7.727260 &     7.723680\\
11  &  7.008712 &     6.315243 &     7.054828\\
12  &  5.211963  &    6.311325   &   4.857754\\
\hline
\end{tabular}
\section{Partdiff-posix im Koordinatensystem}
	\begin{tikzpicture}
		\begin{axis}
		[
			legend entries = 
			{%
			\textbf{Test1},
			\textbf{Test2},
			\textbf{Test3},
			},
			width=0.9\textwidth,height=0.5\textheight,
			ylabel={Sekunden},xlabel={Anzahl der Threads},
		]
			\addplot table[x=Thread, y=Test1] {Data.csv};
			\addplot table[x=Thread, y=Test2] {Data.csv};
			\addplot table[x=Thread, y=Test3] {Data.csv};
		\end{axis}
	\end{tikzpicture}
\section{Interpretation der Ergebnisse}
Die benötigte Zeit sinkt offensichtlich quadratisch (3). Außerdem schwanken die Testwerte
je mehr Threads zum Einsatz kommen (2) und (3), was ungewöhnlich ist, da unser Schedulingsverfahren statisch ist. Man kann außerdem annehmen, dass man anhand der Partdiff-posix mit 12 Threads einen Speedup $\geq$ 8 erhält, da der kleinste Testwert von Partdiff-seq
mit $51.954834$ und der größte Testwert von Partdiff-seq mit $6.311325$ ein Speedup von $\approx 8.232001$ erreicht.


\end{document}