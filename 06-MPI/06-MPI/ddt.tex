\documentclass[a4paper,11pt]{scrartcl}
\usepackage[utf8]{inputenc}
\usepackage{amsmath}
\usepackage{amssymb}
\usepackage{dsfont}
\usepackage{pgfplots}
\pagestyle{empty}
\title{DDT}
\author{Klaus-Johan Ziegert \and Tobias Knoppler}

\begin{document}
\maketitle

\section{}%section name}\label{label}
\begin{itemize}
  \item Wie kann man die Programmparameter angeben?\\
    ~~ Arguments-Textfeld im RUN-Formular (öffnet sich nach dem Starten).
  \item Stepmöglichkeiten:
    \begin{enumerate}
      \item Step into\\
        betritt den aufgerufenen Befehl (führt diese im schrittweisen debugmodus aus).
      \item Step over\\
        Führt den aufgerufenen Befehl aus, ohne innerhalb dessen Stack zu pausieren.
      \item Step out\\
        führt den momentanen Stack vollständig aus und hält beim Elternstack wieder (im debugmodus).
    \end{enumerate}
  \item Werte der Rangvariablen\\
    Es werden teilweise negative Wert angezeigt (uint als int interpretiert?).\\
    Wahrscheinlich Speicheradressen
\end{itemize}

\end{document}
